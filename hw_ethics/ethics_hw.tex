\documentclass[11pt]{article}
\usepackage[utf8]{inputenc}

%%%%%%%%%%%%%%%%%%%%%%%%%%%%%%%%%%%%%%%%%%%%%%%%%%%%%%%%%%%%%%%%
% PACKAGES, FORMATTING, LISTING, AND OTHER GRAPHICAL ELEMENTS  %
%%%%%%%%%%%%%%%%%%%%%%%%%%%%%%%%%%%%%%%%%%%%%%%%%%%%%%%%%%%%%%%%

\usepackage{amsmath,amssymb}
\usepackage[T1]{fontenc} % font encoding
\usepackage{graphicx} % for figures
\usepackage{float} % for figures
\usepackage{indentfirst} % indent first paragraph of section
\usepackage{soul,xcolor} % for colors
\usepackage{xkcdcolors} % colors from XKCD color survey
\usepackage{mathtools} % for boxed equations within \align
\usepackage{sectsty} % adjust section headers
\usepackage{fancyhdr} % page headers
\usepackage{titling} % reference title,author,date (in fancyhdr page header)
\usepackage{breqn} % automatically line wrap long Eqs.
\usepackage{pdfpages} % for \includepdf
\usepackage{pdflscape} % landscape 
\usepackage{mdframed} % for framed environment
\usepackage{listings} % for code blocks
\usepackage{inconsolata} % better monospace font
\usepackage{matlab-prettifier} % Better MATLAB listing highlighting
\usepackage[letterpaper, total={6.5in, 9in}]{geometry} % set paper and margin size
\usepackage{empheq} % multiline boxed equations, etc.
\usepackage{hyperref}
\usepackage{dsfont}	% gives you \mathds{} font
% \usepackage{enumerate} % custom enumerate numbering (or lettering in this case)
\usepackage[shortlabels]{enumitem} % more enumerate options
\usepackage{svg} % use inkscape to import svgs
\usepackage{bm} % bold symbols
\usepackage[parfill]{parskip} % replace indentation with paragraph spacing
\usepackage{dsfont} % gives you \mathds{} font

% hyperbolic trig inverse functions
\DeclareMathOperator{\sech}{sech}
\DeclareMathOperator{\csch}{csch}
\DeclareMathOperator{\arcsec}{arcsec}
\DeclareMathOperator{\arccot}{arcCot}
\DeclareMathOperator{\arccsc}{arcCsc}
\DeclareMathOperator{\arccosh}{arcCosh}
\DeclareMathOperator{\arcsinh}{arcsinh}
\DeclareMathOperator{\arctanh}{arctanh}
\DeclareMathOperator{\arcsech}{arcsech}
\DeclareMathOperator{\arccsch}{arcCsch}
\DeclareMathOperator{\arccoth}{arcCoth} 

% add page headers
\pagestyle{fancy}
\fancyhf{}
\fancyhead[LE,LO]{\thetitle}
\fancyhead[RE,RO]{\theauthor}
\fancyfoot[CE,CO]{\thepage}

% adjust section header font size
\sectionfont{\fontsize{20}{15}\selectfont}
\subsectionfont{\fontsize{14}{15}\selectfont}

% vertical table spacing
\renewcommand{\arraystretch}{1.5}

% right-side cases
% \newenvironment{rcases}
%     {\left.\begin{aligned}}
%     {\end{aligned}\right\rbrace}

% Multi-line \fbox
\newcommand\multilineBox[1]{
\noindent
\fbox{
	\parbox{0.97\linewidth}
		{#1}
	}
}

% number just one equation in an un-numbered align* environment
\newcommand\numberthis{\addtocounter{equation}{1}\tag{\theequation}}

%%%%%%%%%%%%%%%%%%%%%%%%%%%%%%%%%%%%%%%%%%%%%%%%%%%%%%%%%%%%%%%%

% set listings font to inconsolata
\DeclareFixedFont{\ttb}{T1}{txtt}{bx}{n}{9} % for bold
\DeclareFixedFont{\ttm}{T1}{txtt}{m}{n}{9}  % for normal

% general listings environment
\lstnewenvironment{code}
{\lstset{
    frame=single,
    % backgroundcolor=\color{xkcdPale},
    basicstyle=\fontfamily{pcr}\selectfont\tiny
    }}
{}

% MATLAB listings environment
\lstnewenvironment{MATLAB}
{\lstset{
    style=Matlab-editor,
    frame=single,
    % backgroundcolor=\color{xkcdPale},
    basicstyle=\fontfamily{pcr}\selectfont\tiny
    }}
{}

% Python style for listings
\newcommand\pythonstyle{\lstset{
    language=Python,
    basicstyle=\ttm,
    morekeywords={self},              % Add keywords here
    keywordstyle=\ttb\color{xkcdGreen},
    morekeywords=[2]{
        array,
        pi,
        zeros,
        max,
        sin,
        cos,
        linspace,
        size,
        plot,
        xlabel,
        title,
        legend,
        show,
        concatenate,
        logspace,
        exp,
        ylabel,
        savefig,
        grid,
        figure,
        axes
    },
    keywordstyle = [2]{\color{xkcdBrightBlue}},
    emph={__init__},                  % Custom highlighting
    emphstyle=\ttb\color{xkcdRed},    % Custom highlighting style
    stringstyle=\color{xkcdGreen},
    backgroundcolor=\color{xkcdPale},
    numberstyle=\color{xkcdGrey},
}}

% Python listings environment
\lstnewenvironment{python}[1][]
{\pythonstyle \lstset{#1}}
{}

% Python listings for external files
\newcommand\pythonexternal[2][]
{{\pythonstyle \lstinputlisting[#1]{#2}}}

% Python listings inline
\newcommand\pythoninline[1]{{\pythonstyle \lstinline!#1!}}


%%%%%%%%%%%%%%%%%%%%%%%%%%%%%%%%%%%%%%%%%%%%%%%%%%%%%%%%%%%%%%%%
% MATHEMATICAL SHORTCUTS                                       %
%%%%%%%%%%%%%%%%%%%%%%%%%%%%%%%%%%%%%%%%%%%%%%%%%%%%%%%%%%%%%%%%

% real and imaginary components
\renewcommand{\Re}{\operatorname{Re}}
\renewcommand{\Im}{\operatorname{Im}}

% Fourier transforms
\newcommand\fourier[1]{\frac{1}{\sqrt{2\pi}} \int_{-\infty}^\infty #1 e^{-i\omega x} \textrm{d}x}
\newcommand\inverseFourier[1]{\frac{1}{\sqrt{2\pi}} \int_{-\infty}^\infty #1 e^{i\omega x} \textrm{d}\omega}

% derivatives
\newcommand{\ppf}[2]{\frac{\partial #1}{\partial #2}}
\newcommand{\pppf}[3]{\frac{\partial^2 #1}{\partial #2 \partial #3}}
\newcommand{\ddf}[2]{\frac{\text{d} #1}{\text{d} #2}}
\newcommand{\DDf}[2]{\frac{\text{D} #1}{\text{D} #2}}

% norms
\newcommand\norm[1]{\lVert#1\rVert}

% statistical operators
\newcommand{\prob}{\operatorname{P}}
\newcommand{\expectation}{\operatorname{E}}
\newcommand{\variance}{\operatorname{Var}}
\newcommand{\stddev}{\operatorname{SD}}

% statistical distributions
\newcommand{\bernoulli}{\operatorname{Bernoulli}}
\newcommand{\binomial}{\operatorname{Binomial}}
\newcommand{\poisson}{\operatorname{Poisson}}
\newcommand{\normal}{\operatorname{Normal}}

%%%%%%%%%%%%%%%%%%%%%%%%%%%%%%%%%%%%%%%%%%%%%%%%%%%%%%%%%%%%%%%%
% COMMONLY USED COPYPASTAS
%%%%%%%%%%%%%%%%%%%%%%%%%%%%%%%%%%%%%%%%%%%%%%%%%%%%%%%%%%%%%%%%

% CODE BLOCK

% \begin{MATLAB}[caption={MATLAB script},label={code:p1}]
% for i = 1:n
%     disp('string')
% end
% \end{MATLAB}
% Code block \ref{code:p1}



% MULTIPLE EQUATIONS BOXED

% \begin{empheq}[box=\fbox]{align}
% \end{empheq}



% ENUMERTATE WITH a) instead of 1.

% \begin{enumerate}[label=\alph*)]
% \end{enumerate}
% 'Problem' sections
% \renewcommand{\thesection}{Part \arabic{section}}
\renewcommand{\thesubsection}{\arabic{section}.\alph{subsection}}
% \renewcommand{\thesubsection}{\arabic{subsection}}

\title{ENGR 510 Ethics Homework}
\author{Anthony Su}
\date{December 6, 2024}

\begin{document}
\thispagestyle{plain}
\maketitle


%%%%%%%%%%%%%%%%%%%%%%%%%%%%%%%%%%%%%%%%%%%%%%%%%%%%%%%%%%%%%%%%%%%%%%%%%%%%%%%%
% 1. Review or Choose and Article for an Ethics Discussion
%%%%%%%%%%%%%%%%%%%%%%%%%%%%%%%%%%%%%%%%%%%%%%%%%%%%%%%%%%%%%%%%%%%%%%%%%%%%%%%%
\section{Article}

\subsection{} % a --------------------------------------------------------------
Price, I., Sanchez-Gonzalez, A., Alet, F. et al. Probabilistic weather forecasting with machine learning. Nature (2024). https://doi.org/10.1038/s41586-024-08252-9

\subsection{} % b --------------------------------------------------------------
Weather forecasts today rely on ensemble numerical physical simulations. An
important feature of these models is their dual deterministic (realistic and
detailed trajectories) and probabilistic (accurate distributions of
trajectories) accuracy. The authors developed a machine learning model called
GenCast which has these key features and outperforms existing numerical physical
simulations in weather prediction accuracy with reduced computational cost.
GenCast's architecture is based on conditional diffusion.

%%%%%%%%%%%%%%%%%%%%%%%%%%%%%%%%%%%%%%%%%%%%%%%%%%%%%%%%%%%%%%%%%%%%%%%%%%%%%%%%
% 2. Engineering Code of Ethics
%%%%%%%%%%%%%%%%%%%%%%%%%%%%%%%%%%%%%%%%%%%%%%%%%%%%%%%%%%%%%%%%%%%%%%%%%%%%%%%%
\section{Engineering Code of Ethics}

\subsection{} % a --------------------------------------------------------------
I have no experience in weather prediction and only cursory knowledge of machine
learning (and none in diffusion models). I am not qualified to serve weather
forecasts to the public. I would require a team of machine learning engineers
(who understand diffusion models) and meteorologists (who can validate the
models) in order to perform the work.

\subsection{} % b --------------------------------------------------------------
Standards which are imposed on the outputs of the simulations would be
straightforward to apply. For example, data formats, standard statistical
metrics, and empirical validation methods would be the same.

However, standards imposed on the model itself (such as enforcing certain
conservation laws, stability measures, etc.) would be difficult to apply since
these concepts may not be well-defined for the model as it is currently
implemented.

\subsection{} % c --------------------------------------------------------------
The machine learning model in this case is not too unlike the model it is
replacing; it is a probabilistic model. Thus, the usual caveats about
uncertainty and confidence intervals apply.

\subsection{} % d --------------------------------------------------------------
\textbf{Hold paramount the safety, health, and welfare of the public.}

Research and development of predictive machine learning weather prediction
models should always be validated to standard levels before being used to
publish forecasts to the public, even at the cost of research progress. For
example, if research funding to bring a model to a higher TRL is contingent on
accurate published forecasts, the model must be well validated before publishing
these forecasts to the public.

%%%%%%%%%%%%%%%%%%%%%%%%%%%%%%%%%%%%%%%%%%%%%%%%%%%%%%%%%%%%%%%%%%%%%%%%%%%%%%%%
% 3. Benefits and Harms
%%%%%%%%%%%%%%%%%%%%%%%%%%%%%%%%%%%%%%%%%%%%%%%%%%%%%%%%%%%%%%%%%%%%%%%%%%%%%%%%
\section{Benefits and Harms}

\subsection*{Benefits} % Benefits ----------------------------------------------
The immediate benefit from this model is increased accuracy. This will increase
the quality of life (happiness) for all affected as plans based on weather
forecasts are interrupted less frequently and dangerous weather events are
forewarned more consistently.

A secondary benefit is increased equality as the barrier to entry for weather
prediction is reduced. Not only is the cost of weather forecasting reduced, but
also climate science education and research. These models can be used for
acadeemic purposes in lower-income regions, increasing their scientific
and economic autonomy.

\subsection*{Harms} % Harms ----------------------------------------------------
A weather model can exacerbate global inequality. If the training data for the
model was not evenly sampled with uniform precision and accuracy around the
globe, then wealthier regions with a higher density of weather instrumentation
could be more accurately represented in the training data and thus more
accurately modeled by the predictive model.

A weather model can also exacerbate global inequality by construction. GenCast
utilizes an equiangular latitude-longitude grid spatial discretization. This
discretization has increased grid density near the poles and decreased grid
density near the equator. Thus, forecasts near the equator will be lower in
resolution and potentially less accurate. This can be avoided by using a more
uniformly distributed discretization such as an isocahedral grid.

%%%%%%%%%%%%%%%%%%%%%%%%%%%%%%%%%%%%%%%%%%%%%%%%%%%%%%%%%%%%%%%%%%%%%%%%%%%%%%%%
% 3. AI or ML Code of Ethics
%%%%%%%%%%%%%%%%%%%%%%%%%%%%%%%%%%%%%%%%%%%%%%%%%%%%%%%%%%%%%%%%%%%%%%%%%%%%%%%%
\section{AI or ML Code of Ethics}

\subsection{} % a --------------------------------------------------------------
\begin{figure}[H]
    \centering
    \includegraphics[width=6.5in]{BoeingPrinciplesforAIEnabledSystems.pdf}
    \caption{Boeing Principles for AI-Enabled Systems}
    \label{3afig1}
\end{figure}

\subsection{} % b --------------------------------------------------------------
In a few sentences, discuss what you think of the code you found. Some things you might
consider: Does it seem complete to you? Does it seem relevant to the work you do? Is it easy
to interpret? Is it already out of date, or do you anticipate that it will be durable, or quickly
become obsolete?

The principles appear complete and relevant, especially in the context of
engineered systems. However, it does not address the issue of bias which is a
particularly harmful omission when data-driven models are trained on human or
otherwise societal data. While this issue likely does not apply to the most
obvious applications of machine learning at Boeing (e.g. system identification
of engineered systems, flight anomaly detection, manufacturing quality control,
etc.), Boeing products ultimately are operated by, for, and around humans and
human society, and thus there is potential for unintentional reinfocement of
biases.

\end{document}